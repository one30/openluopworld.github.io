% The template is from
% https://www.sharelatex.com/templates/presentations/mauclair's-leanprogress-beamer-theme

% Modified by LuoPeng
% 2017.05.18

\documentclass[xcolor=svgnames,handout]{beamer}

\usepackage[utf8]    {inputenc}
\usepackage[T1]      {fontenc}
\usepackage[english] {babel}

\usepackage{amsmath,amsfonts,graphicx}
\usepackage{beamerleanprogress}
\usepackage{ctex} % for Chinese
\usepackage{verbatim} % for multi line of comments


\title
  []% for sub title
  {密码算法简介}

\author
  [罗鹏]
  %{罗鹏\quad 张洁\quad 李光成\\刘明昊\quad 顾守仁}
  {罗鹏}

\date
  {2017.01.01}

\institute
  {中国科学院信息工程研究所}


\begin{document}

\maketitle

\section{大纲}
\begin{frame}
  {大纲}
  \begin{itemize}
    \item 对称密码算法 (私钥密码算法)
	    \begin{itemize}
	 	\item 分组密码 (Block Cipher)
		\item 流密码 (Stream Cipher)
		\item 工作模式 (Mode of Operation)
		\item 密码杂凑函数 (Hash Function)
		\item 消息认证码 (Message Authentication Code)
		\item 认证加密 (Authenticated Encryption)
	    \end{itemize}
    \item 非对称密码算法 (公钥密码算法)
	    \begin{itemize}
	 	\item 密钥交换算法 (Key Exchange)
		\item 公钥加密 (Public-Key Encryption)
		\item 数字签名 (Digital Signature Algorithm)
    \end{itemize}
  \end{itemize}
\end{frame}

\section{分组密码}
\begin{frame}
  {分组密码 (Block Cipher)}
  \begin{table}
  	\begin{tabular}{cc}
  		\begin{minipage}{.5\linewidth}
  			经典分组密码
  			\begin{itemize}
  				\item \begin{alertenv}\textbf{DES}\end{alertenv}
  				\item \begin{alertenv}\textbf{AES}\end{alertenv}
  				\item \begin{alertenv}\textbf{Serpent}\end{alertenv}
  				\item Twofish
  				\item \begin{alertenv}\textbf{SM4}\end{alertenv}
  				\item RC6
  				\item GOST
  				\item ...
  			\end{itemize}
  		\end{minipage}
  		&
  		\begin{minipage}{.5\linewidth}
  			轻量级分组密码
	  	    \begin{itemize}
	  	    	\item HIGHT
	  			\item \begin{alertenv}\textbf{PRESENT}\end{alertenv}\cite{bogdanov2007present}
	  			\item LBlock
	  			\item PRINCE
	  			\item \begin{alertenv}\textbf{RECTANGLE}\end{alertenv}\cite{zhang2015rectangle}
	  			\item \begin{alertenv}\textbf{SIMON}\end{alertenv}\cite{beaulieu2015simon}
	  			\item SKINNY
	  			\item ...
	  		\end{itemize}
  		\end{minipage}
  	\end{tabular}
  \end{table}
\end{frame}

\section{流密码}
\begin{frame}
	{流密码 (Stream Cipher)}
	\begin{itemize}
		\item RC4
		\item Salsa20
		\item ChaCha
		\item ...
	\end{itemize}
\end{frame}

\section{工作模式}
\begin{frame}
	{工作模式 (Mode of Operation)}
	\begin{itemize}
		\item ECB
		\item \begin{alertenv}\textbf{CBC}\end{alertenv}
		\item CFB
		\item OFB
		\item \begin{alertenv}\textbf{CTR}\end{alertenv}
	\end{itemize}
\end{frame}

\section{密码杂凑函数}
\begin{frame}
	{密码杂凑函数 (Hash Function)}
	\begin{itemize}
		\item \begin{alertenv}\textbf{MD5}\end{alertenv}
		\item \begin{alertenv}\textbf{SHA1}\end{alertenv}
		\item \begin{alertenv}\textbf{SHA2}\end{alertenv}
		\item \begin{alertenv}\textbf{SHA3}\end{alertenv}
		\item BLAKE2
		\item Gr{\o}stl
		\item JH
		\item \begin{alertenv}\textbf{SM3}\end{alertenv}
		\item ...
	\end{itemize}
\end{frame}

\section{消息认证码}
\begin{frame}
	{消息认证码 (Message Authentication Code)}
	\begin{itemize}
		\item CBC-MAC
		\item \begin{alertenv}\textbf{HMAC}\end{alertenv}
		\item OMAC
		\item \begin{alertenv}\textbf{Poly1305}\end{alertenv}
		\item DAA
		\item ...
	\end{itemize}
\end{frame}

\section{认证加密}
\begin{frame}
	{认证加密 (Authenticated Encryption)}
    \begin{table}
    	\begin{tabular}{cc}
    		\begin{minipage}{.5\linewidth}
				\begin{itemize}
					\item CCM
					\item EAX
					\item \begin{alertenv}\textbf{GCM}\end{alertenv}
					\item OCB
					\item ...
				\end{itemize}
			\end{minipage}
			\begin{minipage}{.5\linewidth}
				Caesar Candidates
			    \begin{itemize}
					\item ACORN
					\item Deoxys
					\item PRIMATEs
					\item SHELL
					\item ...
				\end{itemize}
			\end{minipage}
		\end{tabular}
	\end{table}
\end{frame}

\section{密钥交换算法}
\begin{frame}
	{密钥交换算法 (Key Exchange)}
	\begin{itemize}
		\item Diffie–Hellman密钥交换,简称DH
		\item 基于非对称密码算法
		\begin{itemize}
			\item Elliptic curve Diffie–Hellman (ECDH或者ECDHE),基于ECC
		\end{itemize}
	    \item ...
	\end{itemize}
\end{frame}

\section{公钥加密算法}
\begin{frame}
	{公钥加密 (Public-Key Encryption)}
	\begin{itemize}
		\item RSA
		\item ECC
		\item ...
	\end{itemize}
\end{frame}

\section{数字签名}
\begin{frame}
	{数字签名 (Digital Signature Algorithm)}
	数字签名都是通过非对称密码算法实现。
	\begin{itemize}
		\item RSA
		\item Elliptic Curve Digital Signature Algorithm (ECDSA),基于ECC
		\item Edwards-curve Digital Signature Algorithm (EdDSA)
		\item ...
	\end{itemize}
\end{frame}

\begin{comment}
\begin{frame}
	{A Movie}
	\begin{block}{Some block}
		\begin{itemize}
			\item Movies only seem to work in Adobe Reader
			\item Movie file is not embedded, it must be on the computer
		\end{itemize}
	\end{block}
	
	\begin{exampleblock}{Some more block}
		Movies only seem to work in Adobe Reader\par
		Movie file is not embedded, it must be on the computer
	\end{exampleblock}
	
	\begin{alertblock}{}
		Some text in here.
		\begin{itemize}
			\item Movies only seem to work in Adobe Reader
			\item Movie file is not embedded, it must be on the computer
		\end{itemize}
	\end{alertblock}
\end{frame}
\end{comment}

\begin{frame}
	\bibliographystyle{alpha}
	\bibliography{cipher}
\end{frame}

\end{document}

